\documentclass[12pt,a4paper]{article}
\usepackage[linktocpage]{hyperref}
\hyphenation{for-ne-ce}
\renewcommand{\contentsname}{\'Indice}
\begin{document}
\vspace*{6cm}
\huge
\begin{center}
FAQ
Geomview
\end{center}
\normalsize
\newpage
\tableofcontents
\newpage

\section{Quest\~oes Gerais}

    \subsection{O Que \'e Geomview?}

        Um programa visualizador de prop\'{o}sito geral e interativo para Unix. \'E usado
        na maioria das vezes para gr\'aficos tridimensionais mas pode mostrar dados em 2D e 4D tamb\'em. Veja
        a vis\~ao geral  \url{http://www.geomview.org/overview} para coment\'arios mais
        gerais sobre o Geomview.

    \subsection{Como fa\c{c}o para ter o arquivo de instala\c{c}\~ao do Geomview?}

        Geomview est\'a dispon\'ivel livremente a partir de  \url{http://www.geomview.org/download}.
        Existem distribui\c{c}\~oes no formato bin\'ario para m\'aquinas Linux, FreeBSD, SGI, Sun SPARC,
        HP-UX, IBM RS/6000, DEC Alpha, e NeXT, bem como uma distribui\c{c}\~ao na forma de
        c\'{o}digo fonte.

        Voc\^e pode tamb\'em ter o arquivo de instala\c{c}\~ao por ftp an\^onimo a partir de \url{ftp://ftp.geomview.org/pub}.

        Geomview \'e um software livre, mas gostar\'iamos saber das pessoas que o utilizam.
        Por favor envie-nos um email para register@geomview.org nos dizendo o que
        voc\^e est\'a fazendo com o Geomview.

    \subsection{Qual a documenta\c{c}\~ao para o Geomview que est\'a dispon\'ivel?}

        See the Documentation \url{http://www.geomview.org/docs} part of this web site.

    \subsection{Como eu posso interagir com outro usu\'arios do Geomview?}

        Existe a lista de mensagens ``geomview-users'' para pessoas usando geomview que
        pode ser usada para comunica\c{c}\~ao entre usu\'arios sobre problemas do geomview,
        quet\~oes, experi\^encias, etc. Os autores do geomview s\~ao tamb\'em uma parte dessa
        lista e ir\~ao responder a quest\~oes postadas. Tamb\'em usamos essa
        lista para fazer an\'uncios sobre novas vers\~oes e outras coisas do
        interesse dos usu\'arios. Para usar essa lista, envie um email vazio com 'subscribe'
        na linha de assunto para geomview-users-request@lists.sourceforge.net, ou visite a p\'agina
        da lista em:
        \url{http://lists.sourceforge.net/mailman/listinfo/geomview-users}.
	
        Veja tamb\'em a lista de outros software e projetos:
        \url{http://www.geomview.org/thirdparty}.

\section{Configuration/Installation/ Execution Problems}

    \subsection{As caixas de verifica\c{c}\~ao e certos outros recursos da interface gr\'afica de usu\'ario est\~ao ou 
    ausentes dos pain\'eis do Geomview, ou n\~ao trabalham quando eu compilo 
    a \'ultima vers\~ao. O que est\'a acontecendo?}

        Esses problemas parecem estarem associados com vers\~oes mais recentes da Lesstif
        (e.g. 0.91.x), pelo menos no GNU/Linux. N\~ao sei se isso \'e um problema
        com o Lesstif propriamente dito, ou se alguma coisa est\'a errada com o caminho que o Geomview usa
        Lesstif. Enquanto isso, se voc\^e experiencia esse problema, sugiro
        compilar Geomview com a Open Motif ao inv\'es da Lesstif. Existe um
        arquivo leve de distribui\c{c}\~ao no formato bin\'ario da Open Motif 2.1.30 dispon\'ivel na
        pagina de download do Geomview  \url{http://www.geomview.org/download}, com
        instru\c{c}\~oes para us\'a-lo com o Geomview. Ou, voc\^e pode pegar a arquivo completo de distribui\c{c}\~ao da
        Open Motif (fonte ou bin\'ario) a partir de \url{http://www.opengroup.org/motif}
        ou de \url{http://www.openmotif.com}.

        Se voc\^e est\'a familiarizado com Lesstif e conhece o que pode causar ese problema
        (e especialmente se voc\^e sabe como consertar isso!), por favor me mand um email
        (mbp at geomtech.com).

    \subsection{O configure alega n\~ao poder encontrar a OpenGl no meu sistema, mas tenho certeza que ela est\'a instalada}
    \label{aswer:OpenGl}
        \begin{itemize}
          \item Garanta que voc\^e informou o argumento ``$--$with-opengl=DIR'' ao configure,
            onde DIR \'e o diret\'{o}rio contendo sua instala\c{c}\~ao da OpenGL. DIR
            deve ser o caminho absoluto para um diret\'{o}rio contendo subdiret\'{o}rios
            chamados ``include'' e ``lib''. O subdiret\'{o}rio ``include'' deve por sua vez
            incluir um subdiret\'{o}rio chamado ``GL'' que tem o arquivo de cabe\c{c}alho ``gl.h'' (bem
            como outros arqwuivos de cabe\c{c}alho) dentro dele. O diret\'{o}rio ``lib'' deve
            conter os arquivos (.so) da biblioteca GL.

          \item Algumas vezes 'configure' ir\'a incorretamente reportar que a OpenGl est\'a faltando
            quando de OpenGl estiver presente, mas o teste para detectar a presen\c{c}a da OpenGL falhou por alguma
            outra raz\~ao, tal como algum arquivo de cabe\c{c}alho ou biblioteca dependente faltando.
            Por exemplo, se sua instala\c{c}\~ao do X window est\'a danificada ou
            incompleta, isso pode confundir os testes que 'configure' faz para
            OpenGl. Existem dois lugares para olhar para colher ind\'icios a respeito disso:
              \begin{itemize}
              \item A saida do 'configure' propriamente dito --- olhe para as linhas relacionadas
                \`a verifica\c{c}\~ao para X window, em particular. Se o X n\~ao for encontrado, ou
                o Geomview n\~ao puder ser linkado com suporte ao X, ent\~ao esse \'e provavelmente o problema.

              \item O arquivo 'config.log' queno 'configure' escreve enquanto \'e executado. Esse
                arquivo cont\'em todos os detalhes terr\'ivies sobre os testes que
                'configure' est\'a fazendo. O 'config.log' ir\'a ter mensagens de erro que podem
                indicar por que certos testes falham. IMPORTANTE nota sobre a leitura do
                'config.log': muitos dos testes do configure envolvem la\c{c}os que tentam
                muitas possibilidades --- por exemplo muitas localiza\c{c}\~oes poss\'iveis
                para um arquivo de cabe\c{c}alho. O 'configure' ir\'a escrever um pequeno programa e
                tentar compl\'a-lo uma vez para cada uma dessas localiza\c{c}\~oes, at\'e enconrar
                encontrar uma que trabalhe. Paa cada uma que n\~ao funcionou, ir\~ao existir
                mensagens de erro no arquivo 'config.log'. Quando for ler o
                'config.log', garanta olhar para todos esses testes, n\~ao apenas o
                primeiro deles, durante a tentativa de decidor por que um teste est\'a falhando.
             \end{itemize}
          \item se a sa\'ida do 'configure', ou o conte\'udo do 'config.log',
            sugerem que alguma parte do X n\~ao pode ser encontrada (por exemplo se n\~ao puder
            encontrar certos cabe\c{c}alhos do X, como o ``X11/X.h'' ou o $----$\newline ``X11/Intrinsic.h''),
            ent\~ao o problema pode ser que voc\^e n\~ao tenha instalado o pacote
            de desenvolvimento do X window para o seu sistema. Alguma distribui\c{c}\~ao
            GNU/Linux inclue o pacote com os execut\'aveis mas n\~ao o(s) pacote(s) de
            desenvolvimento. Garanta ter instalado quaisquer pacotes que forem
            necess\'arios para trabalho de desenvolvimento do X bem como o(s) pacote(s) com o(s) execut\'avei(s) do X.

          \item Se 'configure' reclama que n\~ao pode linkar com -lGL (ou -lGLU) mas
            voc\^e tem certeza que eles est\~ao a\'i, encontre os diret\'{o}rios contendo seus
            arquivos libGL.so.* e libGLU.so.*; chame esse diret\'{o}rio de DIR. DIR ir\'a
            provavelmente conter um ou mais arquivos com nomes como libGL.so.VERS\~AO
            e libGLU.so.VERS\~AO, onde VERS\~AO is some version number,
            such as ``1.2.0'' or ``1.2.030200''. Tamb\'em deve conter entradas
            chamadas simplesmente libGL.so e libGLU.so, sem sfixo de VERS\~AO; esses
            s\~ao comumente links simb\'{o}licos aos arquivos correspondentes com sufixos de
            vers\~ao. Por exemplo, no meu sistema tenho
	    \begin{verbatim}
            libGL.so -> libGL.so.1.2.030200
            libGLU.so -> libGLU.so.1.2.030200
	    \end{verbatim}
            Se os links (ou arquivos) libGL.so e libGLU.so n\~ao estiverem presentes,
            crie-os fazendoos apontar para os arquivos correspondentes com
            o n\'umero mais alto de vers\~ao.

            N\~ao entendo por que esses links podem n\~ao serem criados em alguma
            instala\c{c}\~ao da OpenGL, pelo fato de ser do meu entendimento que eles
            terem de estarem a\'i para que os programas possam us\'a-la propriamente. Admito todavia que
            n\~ao entendo todo o stuff .so e .so.VERS\~AO, de forma que pode ser
            que esses links n\~ao sejam realmente necess\'arios e que alguma modifica\c{c}\~ao no
            script 'configure'do Geomview ou nos arquivos Makefiles possam eliminar a necessidade
            deles. Se voc\^e sabe uma forma de fazer isso, por favor me diga
            (mbp@geomtech.com).
     \end{itemize}

    \subsection{O configure alega n\~ao poder encontrar a Motif (ou o Lesstif ou o OpenMotif) no meu sistema, mas tenh certeza que ele est\'a instalado}

        Leia todas as sugest\~oes acima na resposta da quet\~ao an\'aloga
        sobre a OpenGL na se\c{c}\~ao \ref{aswer:OpenGl}; elas todas se aplicam igualmente \`a
        Motif. (O arquivo de cabe\c{c}alho principal a procurar no diret\'{o}rio ``include'' \'e o
        ``Xm/Xm.h'').

\section{Platforms}

    \subsection{Quais plataformas possuem bin\'arios dispon\'iveis \newline para download?}

        SGI Irix, Linux, FreeBSD, Solaris, SunOS, HP, IBM RS/6000, DEC Alpha

    \subsection{N\~ao existe um bin\'ario para minha esta\c{c}\~ao de trabalho. H\'a esperan\c{c}a?}

        Certamente. Se sua esta\c{c}\~ao de trabalho tem um X Window System, OpenGL, Motif,
        e um compilador ANSI (ISO) C, voc\^e pode compilar o geomview a partir da distribui\c{c}\~ao no
        formato de c\'{o}digo fonte em 
        \url{http://www.geomview.org/download}.

        Note que existe uma vers\~ao livre da OpenGL chamada Mesa.
        (\url{http://www.mesa3d.org})
        que executa em programas na maioria dos Unixes livres.
        Veja aquela p\'agina para detalhes sobre os cont\'inuos esfor\c{c}os para incorporar
        suporte a hardwares para algumas das placas gr\'aficas populares.

        Note tamb\'em que existe uma vers\~ao livre da Motif chamada lesstif 
        \url{http://www.lesstif.org/}.

        O arquivo INSTALL \url{http://www.geomview.org/docs/INSTALL} tem instru\c{c}\~oes
        sobre como portar a lesstif para novas arquiteturas. Se voc\^e tiver problemas, envie um email para
        software@geomview.org. Se voc\^e tiver sucesso,
        apreciaremos receber uma c\'{o}pia do seu ``makefile/mk.oquefor'' e
        apender que modifica\c{c}òes de c\'{o}digo fonte foram necess\'arias. Gostar\'iamos tamb\'em
        de incluir seus bin\'arios na nossa lista de distribui\c{c}\~oes pr\'e-compildas.

    \subsection{Por que n\~ao existe uma vers\~ao para Windows?}

        N\~ao existe uma vers\~ao nativa do Geomview para o Microsoft Windows. A
        principal raz\~ao para isso \'e que na \'epoca que o Geomview foi escrito,
        cmputadores pessoais n\~ao eram r\'apidos o suficiente para construir gr\'aficos tridimensionais interativos
        pr\'aticos de forma que focamos nossos esfor\c{c}os em esta\c{c}òes de trabalho Unix. Com o tempo
        PCs r\'apidos o sficiente chegaram, o Geometry Center, onde Geomview foi
        desenvolvido, estava em processo de fechamento. A equipe iniciou o trabalho sobre uma
        vers\~ao para Windows mas n\~ao teve tempo de termin\'a-la antes do Centro
        fechar.

        Geomview pode executar sob Cygwin \url{http://www.cygwin.com}, que fornece
        ao Windows um ambiente semelhante ao Unix. Veja ``Geomview para Windows?''
        \url{http://www.geomview.org/windows/} para maiores informa\c{c}\~oes.

        Se voc\^e gostaria de ver uma vers\~ao do Geomview para Windows, voc\^e pode
        contrbuir para seu desenvolvimento de muitas maneiras. Veja contribuindo para o
        Geomview \url{http://www.geomview.org/contributing} para detalhes.

    \subsection{Tenho acesso a um X11 e a uma esta\c{c}\~ao de trabalho SGI. Qual vers\~ao devo usar?}

        A vers\~ao SGI ir\'a na maioria das vezes sempre ser significantemente mais r\'apida, devido ao
        suporte a hardware para gr\'aficos tridimensionais. Por exemplo, Um Sun Sparcstation 10 \'e
        mais lento que um Indy (antiga entrada de SGI a n\'ivel de m\'aquina). No futuro pode
        ser que exista suporte a hardware para certas placas gr\'aficas OpenGL dispon\'iveis para
        alguns PC Unixes.

    \subsection{O que ocorreu \`a vers\~ao NeXT Quick Renderman?}

        N\~ao mais distribuimos a vers\~ao NeXTStep/OpenStep do Geomview, que
        usou a biblioteca gr\'afica Quick Renderman. Fizemos isso apenas para simplificar
        a manutens\~ao doc\'{o}digo b\'asico ap\'{o}s a vers\~ao 1.5.0. Bin\'arios grandes para as arquiteturas
        Motorola, Intel e HP-PA para a vers\~ao 1.5.0 est\~ao ainda dispon\'iveis em:
        \url{http://www.geomview.org/download/dist/geomview-1.5.0-next.tar}.

    \subsection{What modules are shipped for which platforms with the current release?}

        Liberamos a maioria de todos os m\'{o}dulos para todas as plataformas. A lista de
        m\'{o}dulos distribu\'idos est\'a no arquivo README inclu\'idos nas distribui\c{c}\~oes.
        Se o m\'{o}dulo que voc\^e deseja est\'a naquela lista mas n\~ao aparece na lista de
        m\'{o}dulos no painel principal, Geomview provavelmente n\~ao est\'a instalado de forma correta. Note
        que existem m\'{o}dulos adicionais escritos por terceiros
        \url{http://www.geomview.org/thirdparty} que n\~ao s\~ao parte da distribui\c{c}\~ao
        principal. Os m\'{o}dulos suportados na mais recente vers\~ao para GNU/Linux do
        Geomview (1.9.4) s\~ao:
        \begin{center}
        \begin{tabular}{|l|p{8cm}|}
        \\ \hline
        M\'ODULO & DESCRI\c{C}\~AO  \\ \hline
        Animator & percorre uma sequ\^encia de objetos  \\ \hline
        Antiprism models & Cria, transforma, analisa, e visualiza poliedros \\ \hline
        StageTools & CenterStage, StageManager, StageStills, StageHand - permite criar objetos do Geomview suando f\'{o}rmulas matem\'aticas  \\ \hline
        Clipboard & grava um \'unico objeto OOGL para uma \'area de transfer\^encia  \\ \hline
        Clock & Mostra um rel\'{o}gio anal\'{o}gico na tela  \\ \hline
        Draw Boundary &  \\ \hline
        Nose & depura\c{c}\~ao/exemplo para sele\c{c}\~ao (veja o manual do Geomview)  \\ \hline
        Orrery & Visualiza\c{c}\~ao do Sistema Solar  \\ \hline
        \end{tabular}
	\end{center}
        Os seguintes m\'{o}dulos usam tcl/tk:
        \begin{center}
        \begin{tabular}{|l|p{8cm}|}
        \\ \hline
        M\'ODULO & DESCRI\c{C}\~AO  \\ \hline
        StageTools & CenterStage, StageManager, StageStills, StageHand - permite criar objetos do Geomview suando f\'{o}rmulas matem\'aticas  \\ \hline
        \end{tabular}
	\end{center}
        Os seguintes programas utilit\'arios s\~ao tamb\'em inclu\'idos na distribui\c{c}\~ao
        \footnote{Nota do tradutor: para obter informa\c{c}\~oes adicionais sobre os utilit\'arios abaixo digite
        ``man utilitario'' em uma janela de shell ou ``utilitario $--$help''.}:
	
        \begin{center}
        \begin{tabular}{|l|p{8cm}|}
        \\ \hline
        UTILIT\'ARIO & DESCRI\c{C}\~AO  \\ \hline
        anytooff & convert one or many OOGL files into a single OFF file  \\ \hline
        anytoucd & convert an OOGL file to UCD (AVS) format  \\ \hline
        bdy & compute boundary of an object (helper for drawbdy)  \\ \hline
        bez2mesh & dice BEZ file to list of MESHes  \\ \hline
        clip & clip objects against plane/sphere/cylinder (helper for ginsu)  \\ \hline
        fd2ps & xforms  \\ \hline
        fdesign & xforms  \\ \hline
        hvectext & generate vector text object  \\ \hline
        math2oogl & convert Mathematica graphics to OOGL (helper for OOGL.m)  \\ \hline
        offconsol & polylist vertex consolidator  \\ \hline
        oogl2rib & convert OOGL to Renderman RIB format  \\ \hline
        oogl2vrml & convert OOGL to VRML 1.0  \\ \hline
        oogl2vrml2 &   \\ \hline
        polymerge & merge degenerate OFF vertices/edges/faces (to Evolver or OFF)  \\ \hline
        remotegv &  remotegv $--$help  \\ \hline
        togeomview & send commands to geomview  \\ \hline
        ucdtooff & convert UCD (AVS) format to OFF format  \\ \hline
        vrml2oogl & convert VRML 1.0 to OOGL  \\ \hline
        \end{tabular}
	\end{center}

\section{Usando o Geomview}

    \subsection{Por que os objetos n\~ao aparecem nos lugares corretos?}

        Quando objetos n\~ao aparece onde voc\^e pensa que eles deveriam estar, \'e provavelmente
        devido \`a normaliza\c{c}\~ao habilitada por padr\~ao. Normaliza\c{c}\~ao simplesmente ajusta proporcionalmente um
        a caixa associada de um objeto para ser cotida dentro de uma esfera unit\'aria, com o centro da
        caixa associada transladado para a or\'igem. Esse ajuste \'e \'util quando examinamos um
        objeto simples, e faz voc\^e visualizar o objeto completo sem ter de se
        preocupar com o quanto grande esse objeto \'e. Todavia, isso tamb\'em pode significar que se voc\^e estiver chamando
        m\'ultiplos objetos que s\~ao supostamente pertencentes ao mesmo sistema de
        coordenadas, todos os objetos ir\~ao ser ajustados proporcionalmente e colocados na or\'igem. Para desabilitar
        a normaliza\c{c}\~ao, exponha o painel de apar\^encia. Os controles da
        normaliza\c{c}\~ao est\~ao no quadrante inferior direito do painel. Selecione a op\c{c}\~ao
        ``None''. A tecla de atalho alternativa \'e '0N'\footnote{Nota do tradutor: testei aqui [oO0]N e todas funcionaram.}.

        Para desabilitar a normaliza\c{c}\~ao sempre, personalize o Geomview
        como mostrado em \url{http://www.geomview.org/docs/html/Customization.html}
        inserindo a linha (normalization allgeoms
        none) no arquivo chamado .geomview no seu diret\'{o}rio pessoal de usu\'ario.

        Quando voc\^e desabilitar a normaliza\c{c}\~ao seus objetos podem desaparece. Isso ocorre
        pelo fato do objeto n\~ao normalizado n\~ao cair no cone de visualiza\c{c}\`ao da
        c\^amera. A maneira mais f\'acil de ver tudo \'e escolher o objeto mundo ``World''
        no navegador de objetos, ent\~ao clicar em olhar para ``Look At'' no painel de ferramentas.

    \subsection{Por que est\'a tudo centralizado e/ou uns por cima dos outros?}

        Veja a resposta anterior.

    \subsection{Como posso mostrar uma cole\c{c}\~ao de pontos?}

        A forma mais eficiente de mostrar pontos no Geomview \'e usar o formato de
        arquivo VECT. Esse formato de arquivo \'e principalmente usado para contruir figuras compostas
        por segmentos mas podemos tamb\'em usar esse formato de arquivo para especificar segmentos que possuem somente
        um v\'ertice (i.e. pontos). Vamos dar uma olhada em um arquivo VECT exemplo que
        descreve 3 pontos coloridos em vermelho, verde e azul:
	\tiny
        \begin{verbatim}
        VECT
        3 3 3      # n\'umero de linhas poligonais,
                   # n\'umero de v\'ertices,
                   # n\'umero de cores.

        1 1 1      # a poligonal n\'umero 1 tem 1 v\'ertice,
                   # a poligonal n\'umero 2 tem 1 v\'ertice,
                   # a poligonal n\'umero 3 tem 1 v\'ertice,
                   # nesse caso somente um por estarmos fazendo pontos.

        1 1 1      # a poligonal n\'umero 1 tem 1 cor
                   # a poligonal n\'umero 2 tem 1 cor
                   # a poligonal n\'umero 3 tem 1 cor

        -1 -.2 0   # As coordenadas do \'unico v\'ertice da poligonal 1.
         1 -.2 0   # As coordenadas do \'unico v\'ertice da poligonal 2.
         0  .9 0   # As coordenadas do \'unico v\'ertice da poligonal 3.

        1 0 0 1    # A cor da poligonal 1 no formato RGBA.
        0 1 0 1    # A cor da poligonal 2 no formato RGBA.
        0 0 1 1    # A cor da poligonal 3 no formato RGBA.
        \end{verbatim}
        \normalsize
        Quando chamarmos esse arquivo no Geomview, voc\^e ir\'a provavelmente precisar desabilitar
        a caixa associada (via painel de apar\^encia), caso contr\'ario voc\^e pode n\~ao
        estar apto a ver os pontos.

    \subsection{Como posso tornar os pontos mais facilmente \newline vis\'iveis?}

        Por padr\~ao, a espessura de linhas e pontos no Geomview \'e 1. Essa espessura pode
        estar ok para a maioria das linhas, mas faz com que cada ponto ocupe somente uma
        pixel na tela do computador. Voc\^e pode mudar a espessura das linhas e dos pontos
        adicionando um r\'{o}tulo de apar\^encia no topo de seu arquivo geom\'etrico que se parece
        com isso:

        appearance {
            linewidth 4.
        }

        Nesse caso, temos aumentado nosso tamanho de linha/ponto para 4 e quaisquer pontos
        que tivermos em nosso arquivo ir\'a agora aparecer como pequenos discos. Voc\^e pode tamb\'em mudar
        a largura da linha usando o painel Appearance. O que Geomview faz atualmente
        \'e tratar cada ponto como um pol\'igono de muitos lados que aproxima-se de um disco.

        Se voc\^e deseja que os pontos como objetos s\'{o}lidos tridimensionais, tais como
        pequenas esferas, voc\^e pode usar um m\'etodo completamente diferente para represent\'a-
        los: um objeto INST com m\'ultiplas transforma\c{c}\~oes. Isso permite a voc\^e especificar uma
        apar\^encia geom\'etrica arbitr\'aria para ser usada para representar pontos. Por
        exemplo, o seginte arquivo representa os tr\^es pontos (1.5, 2.0, 0.1),
        (1.0, 0.5, 0.2), e (0.5, 0.3, 0.2) usando pequenos cubos:
        \begin{verbatim}
        INST
        geom {
          OFF
          8 6 12              # VFA
          -0.05 -0.05 -0.05   # V0
           0.05 -0.05 -0.05   # V1
           0.05  0.05 -0.05   # V2
          -0.05  0.05 -0.05   # V3
          -0.05 -0.05  0.05   # V4
           0.05 -0.05  0.05   # V5
           0.05  0.05  0.05   # V6
          -0.05  0.05  0.05   # V7
          4 0 1 2 3           # F0
          4 4 5 6 7           # F1
          4 2 3 7 6           # F2
          4 0 1 5 4           # F3
          4 0 4 7 3           # F4
          4 1 2 6 5           # F5
        }
        transforms
        1 0 0 0  0 1 0 0  0 0 1 0  1.5 2.0 0.1 1
        1 0 0 0  0 1 0 0  0 0 1 0  1.0 0.5 0.2 1
        1 0 0 0  0 1 0 0  0 0 1 0  0.5 0.3 0.2 1
        #
        # these are the matrices:
        #
        # 1   0   0   0     1   0   0   0     1   0   0   0
        # 0   1   0   0     0   1   0   0     0   1   0   0
        # 0   0   1   0     0   0   1   0     0   0   1   0
        # 1.5 2.0 0.1 1     1.0 0.5 0.2 1     0.5 0.3 0.2 1
        \end{verbatim}
        O objeto OFF entre ``geom \{'' e ``\}'' \'e o cubo. As tr\^es linhas
        ap\'{o}s a palavra ``transforms'' are 4x4 transforms, one for each point. Note
        that you can use any valid OOGL expression for the geometry; for
        example, if you want to use small dodecahedra to represent points, you
        could repace the above OFF object with the following, which references
        the dodecahedron object in the file dodec.off (distributed with
        Geomview), scaling it by 0.05:
        \begin{verbatim}
        INST
        geom {
          INST
          geom { < dodec.off }
          transform
            .05   0   0   0
              0 .05   0   0
              0   0 .05   0
              0   0   0   1
        }
        transforms
        1 0 0 0 0 1 0 0 0 0 1 0    1.5 2.0 0.1  1
        1 0 0 0 0 1 0 0 0 0 1 0    1.0 0.5 0.7  1
        1 0 0 0 0 1 0 0 0 0 1 0    0.5 0.3 0.2  1
        \end{verbatim}

        Esteja ciente de que quanto mais complicada geometria voc\^e usa para seus pontos,
        mais tempo ir\'a demorar o Geomview para atualizar a janela. Isso pode ser
        importante se voc\^e est\'a pensando em trabalhar com um grande n\'umero de pontos, nesse caso
        voc\^e deve conduzir-se na dire\c{c}\~ao de apar\^encias de ponto muito simples o usar o m\'etodo de
        mostrar pontos no formato VECT.

    \subsection{Como posso colocar texto em uma cena?}

        Voc\^e tem duas op\c{c}\~oes:
          \begin{itemize}
          \item Voc\^e pode usar o m\'{o}dulo externo Labeler (rotulador-vers\~ao SGI), que fornece a voc\^e uma GUI para
            digitar texto e selecionar fonte: ou vetor ou uma vers\~ao
            poligonalizada de uma fonte instalada. Todavia, voc\^e precisa posicionar o texto em uma cena
            tridimensional, ou manualmente ou com algum outro m\'{o}dulo como o m\'{o}dulo
            Transformer (SGI).
          \item Voc\^e pode usar o programa utilit\'ario de linha de comando hvectext de fones vetoriais de
            Hershey, que permite a voc\^e especificar uma posi\c{c}\~ao para o texto.
            Feito isso voc\^e deve chamar o arquivo resultante no Geomview.
          \end{itemize}
        Se voc\^e n\~ao precisa que o texto seja um objeto tridimensional na cena, voc\^e pode
        criar uma imagem \footnote{\url{http://www.geomview.org/FAQ/answers.shtml\#images}.}
        ou um arquivo postscript \footnote{\url{http://www.geomview.org/FAQ/answers.shtml\#ps}}
        da cena e ent\~ao usar um editor de imagens como o ''Illustrator",
        Showcase, ou o XPaint\footnote{Nota do tradutor: temos tamb\'em o gimp e o xfig.} para colocar o texto.

    \subsection{O Geomview faz visualiza\c{c}\~ao de volume?}

        N\~ao, Geomview \'e pensado para fazer visualiza\c{c}\~ao de superf\'icies. Voc\^e pode ou
        criar uma isosuperf\'icie e ent\~ao v\^e-la usando o Geomview, ou usar umpacote de
        visualiza\c{c}\~ao de volume. O toolkit vtk livre \url{http://www.vtk.org/}
        de visualiza\c{c}\~ao tem suporte extensivo para visualiza\c{c}\~ao de volume, como faz tamb\'em
        os pacotes comerciais como AVS \url{http://www.avs.com}, Iris Explorer
        \url{http://www.nag.co.uk/Welcome\_IEC.html}, or IBM Data Explorer
        . \url{http://pic.dhe.ibm.com/infocenter/dataexpl/v8r2/index.jsp}. \,\,\,\,\,\,\,\,O Volvis
        \url{http://labs.cs.sunysb.edu/labs/vislab/volvis/} \'e um software livre especificamente para
        visualiza\c{c}\~ao de volume.

    \subsection{Pode o geomview fazer mapas de textura?}

        Sim, no release 1.6 e mais novos, mas somente na vers\~ao OpenGL, n\~ao na
        vers\~ao X11.

    \subsection{Por que Geomview n\~ao l\^e meu arquivo OFF?}

        Isso \'e devido provavelmente um interpreta\c{c}\~ao diferente de como um OFF deve
        ser escrito. Geomview indexa v\'ertices iniciando em zero, enquanto alguns outros
        programas comprovadamente iniciam em um. O seguinte programa na linguagem C ir\'a converter
        um arquivo OFF no formato texto puro indexado a partir da unidade em um arquivo OFF indexado a partir do zero.
        \footnote{Nota do tradutor: digamos que voc\^e compile o programa com o comando
        ``gcc 01.c -o 01''. Para usar o programa acima em um shell fa\c{c}a ``01 $<$ arquivo1.off $>$ arquivo0.off''.}
        \begin{verbatim}
        #include <stdio.h>
        #include <string.h>
        #include <stdlib.h>
        int main(void) {
            char s[256];
            int v, f, i, n, t;
            gets(s);
            if (strcmp(s, "OFF")) {
                fprintf(stderr, "not an OFF\n");
                exit(1);
            }
            puts(s); gets(s); puts(s);
            sscanf(s, "%d %d %d", &v, &f, &i);
            for (i=0; i!=v; ) {
                gets(s);
                if (strlen(s)) {
                    puts(s); i++;
                }
            }
            for (i=0; i!=f; i++) {
                scanf("%d", &n);
                printf("\n%d", n);
                for (v=0; v!=n; v++) {
                    scanf("%d", &t);
                    printf(" %d", t-1);
                }
            }
            printf("\n");
            return 0;
        }
        \end{verbatim}

    \subsection{Como posso colocar movimento em uma sequ\^encia de arquivos do Geomview/OOGL?}

        Voc\^e pode tentar usando o Animator, um m\'{o}dulo externo que \'e distribu\'ido
        com todas as vers\~oes do Geomview. Com Animator, voc\^e pode dizer ao Geomview para
        ler uma sequ\^encia de arquivos OOGL e ent\~ao exibir essa sequ\^encia
        avan\c{c}ando, voltando e tamb\'em em passo de quadro usando interfaces semelhantes ao
        VCR\footnote{Nota do tradutor: o VCR virou avifile \url{http://avifile.sourceforge.net/}.
        Temos tamb\'em o kino e o muan.}.

        Para usar o Animator clique na entrada Animator no navegador de m\'{o}dulos externos
        do Geomview. Se o Animator n\~ao aparecer no navegador, ent\~ao o Geomview
        provavelmente n\~ao foi instalado adequadamente. Para mais informa\c{c}\~ao sobre
        Animator leia o painel info dispon\'ivel atrav\'es do programa ou a
        p\'agina de manual (digitando ``man animate'').

\section{Sa\'ida}

    \subsection{Como posso criar uma anima\c{c}\~ao de v\'ideo (MPEG/ QuickTime/animated GIF)?}

Existem muitas variantes dessa quet\~ao:

        \begin{itemize}

        \item primeira variante 
    	    \tiny
    	    \begin{verbatim}
            > Gostaria de gravar uma sequ\^encia de arquivos instant\^aneso no formato ppm de um \'unico
            > objeto no formato off enquanto esse objeto est\'a sendo rotacionado para converter a sequ\^encia
            > em um filme. O \'unico m\'etodo que conhe\c{c}o \'e rotacionar o objeto
            > lentamente com o mouse, parar omovimento, e gravar cada quadro
            > individualmente. Existe um m\'etodo mais r\'apido e autom\'atico, tal como
            > um acript de comandos. Se existe, voc\^e tem um exemplo de script de comando que possa
            > ser modificado?
	    \end{verbatim}
	    \normalsize
            Duas op\c{c}\~oes:
              \begin{itemize}
              \item Se o movimento \'e alinhado a algum eixo, \'e bastante f\'acil usar os
              comandos rotate e snapshot da GCL juntos:
\begin{verbatim}
(snapshot targetcam /tmp/foo\%03d.rgb)
(transform world world world rotate .1 0 0)
(snapshot targetcam /tmp/foo\%03d.rgb)
(transform world world world rotate .1 0 0)
\end{verbatim}
                e assim por diante. O comando snapshot auto-incrementa o nome do arquivo.

              \item Mas para um movimento mais complexo que a simples rota\c{c}\~ao em torno
                do eixo x que mostrei acima, considere usar StageTools, que
                \'e um conjunto de ferramentas pensado para ajudar pessoas a fazerem facilmente
                anima\c{c}\~oes a partir do Geomview. StageTools est\'a inclu\'ido como um m\'{o}dulo nas
                vers\~oes recentes, mas se voc\^e precisar copi\'a-lo ele est\'a dispon\'ivel
                em  \url{http://www.geom.umn.edu/software/StageTools/}.
              \end{itemize}

        \item segunda variante 
    	    \tiny
	    \begin{verbatim}
            > Tenho usado Geomview para ver filmes com ferramentas de anima\c{c}\~ao. como posso
            > converter aquele filme para ooutro formato de anima\c{c}\~ao (e.g. um GIF
            > animado) de forma que possa coloc\'a-lo em uma web page, vis\'ivel para
            > algu\'em sem Geomview?
	    \end{verbatim}
	    \normalsize
	    
            \'E verdade que StageTools ir\'a fazer isso e muito mais tamb\'em. Mas
            existe tamb\'em um caminho muito f\'acil de fazer isso diretamente dentro do m\'{o}dulo
            Animator: a fun\c{c}\~ao Command ir\'a executar um comando GCL ap\'{o}s
            cada quadro. Ent\~ao para pegar automaticamente instant\^aneos a cada quadro, voc\^e dever\'a
            pressionar o bot\~ao Command e digitar alguma coisa como
\begin{verbatim}
(snapshot c0 /tmp/foo\%03d.rgb)
\end{verbatim}
            no campo de texto. Ent\~ao quando voc\^e pressionar play voc\^e ir\'a ver que agora
            agita\c{c}\~ao uma vez que o Animator \'est\'a gravando uma imagem offf para o disco de cada vez. Voc\^e pode
            desejar habilitar o bot\~ao de r\'adio ``Once'' de forma que o Animator pare ap\'{o}s
            executar cada quadro uma \'unica vez. Ent\~ao voc\^e pode usar seu programa
            escolhido para criar um gif animado ou filme do quicktime a partir desso pacote
            de arquivos de imagem. Por exemplo, nos SGIs voc\^e pode fazer isso com o
            ``mediaconvert''.\footnote{Nota do tradutor: temos tamb\'em o recordmydesktop e o kino em GNU/Linux.}
        \end{itemize}

    \subsection{Como posso gravar uma figura de exatamente o que vejo em uma janela de c\^amera?}

        Garanta que a janela de c\^amera que voc\^e deseja seja a janela ativa no momento da foto, ent\~ao selecione
        o item ``Save'' do menu ``File'' no painel principal (ou use a tecla de atalho
        ``$>$''). No painel que aparece, existe uma caixa de eop\c{c}\~ao que \'e ajustada para
        Command por padr\~ao. Selecione uma das op\c{c}\~oes de instant\^aneo, insira o
        nome do arquivo na entrada Selection, e clique ``OK''.

        Na vers\~ao SGI, voc\^e tem tr\^es escolhas de instant\^aneo de imagem: SGI screen,
        PPM screen, e PPM software. Ambas as op\c{c}\~oes screen literalmente gravam os
        os pixels de tela em um arquivo, ou no formato SGI (tamb\'em chamado RGB) ou em formato PPM. A
        op\c{c}\~ao PPM software ir\'a reler a imagem em um espa\c{c}o tempor\'ario de armazenagem fora da tela
        usando o software de releitura a partir da vers\~ao X vanilla do Geomview.
        Dessa forma, pode n\~ao ser id\^entica pixel a pixel a imagem gravada em rela\c{c}\~ao \`a imagem que voc\^e v\^e.

        Na vers\~ao X11, voc\^e tem somente as op\c{c}\~oes PPM.

    \subsection{Como posso fazer um verdadeiro arquivo PostScript que seja bom em m\'ultiplas resolu\c{c}\~oes ao inv\'es de apenas converter um bitmap em um PostScript?}

        Garanta que a janela de c\^amera que voc\^e deseja seja a janela ativa no momento da foto, ent\~ao selecione
        o item ``Save'' do menu ``File'' no painel principal (ou use a tecla de atalho
        ``$>$''). No painel que aparece, existe uma caixa de eop\c{c}\~ao que \'e ajustada para
        Command por padr\~ao. Selecione a op\c{c}\~ao PostScript snapshot, insira o
        nome de arquivo na entrada Selection, e clique ``OK''.

        Esse m\'etodo tem vantagens e desvantagens, comparado ao m\'etodo de gravar um
        bitmap de imagem. A vantagem \'e que o resultado \'e independente da resolu\c{c}\~ao
        -- voc\^e pode imprimir em uma impressora de alta resolu\c{c}\~ao e n\~ao ver qualquer aresta
        quadriculada. As desvantagens s\~ao que nosso releitor PostScript n\~ao pode fazer
        sombreamento suave e usa o algor\'itmo de painter para remo\c{c}\~ao de superf\'icies
        escondidas. Essa \'ultima observa\c{c}\~ao significa que objetos que se interseptam e algumas outras
        cenas de condicionamento host\'il ir\~ao ser desenhadas incorretamente, ou mesmo que objetos
        pr\'{o}ximos ir\~ao ser desenhados atr\'as de objetos distantes. Muitas vezes funciona, mas n\~ao
        sempre.

    \subsection{Por que meu instant\^aneo PostScript est\'a p\'essimo?}

        Veja a resposta anterior.

    \subsection{Como fazer uma imagem de alta qualidade com RenderMan?}

        Se voc\^e tem o Photorealistic Renderman (um prodto comercial da Pixar), ou
        o BMRT (Blue Moon Rendering Toolkit, uma implementa\c{c}\~ao de dom\'inio p\'ublico), voc\^e
        pode criar imagens de alta qualidade com transpar\^encia e ilumina\c{c}\~ao
        mais precisa nas vers\~oes SGI e X11. Paa fazer isso, traga o painel Save
        e selecione ``RMan [->tiff]'' a partir das op\c{c}\~oes de gravar. Insira um nome de arquivo
        e clique  ``Ok''. Aparecer\'a um janela de shell e mude o diret\'{o}rio para onde
        voc\^e gravou o arquivo. Digite ``render /nomearquivo/'' (onde /nomearquivo/ \'e o nome
        do arquivo que voc\^e gravou). Quando terminar, voc\^e ir\'a ter uma imagem de alta qualidade
        em ``/nomearquivo/.tiff''. Para criar uma imagm de alta resolu\c{c}\~ao (para
        reduzir arestas quadriculadas), edite o arquivo que voc\^e gravou. Deve existir uma linha
        adiante das quinze linhas a partir do topo que inicia-se com ``Format'', i.e.
        ``Format 450 450 1''. Os primeiros dois n\'umeros s\~ao a resolu\c{c}\~ao da
        imagem criada. Modifique esses dois n\'umeros para aqueles que voc\^e gostaria (voc\^e deve manter a raz\~ao
        enre os n\'umeros originais para evitar distor\c{c}\~ao), ent\~ao reler o arquivo novamente.

\section{Quest\~oes espec\'ificas do X}

    \subsection{Como posso aumentar a velocidade da vers\~ao X11?}

        Veja a discurs\~ao de op\c{c}\~oes de renderiza\c{c}\~ao na quest\~ao seguinte.

    \subsection{O que fazem os controles Z-Buffer e Dithering no painel de c\^ameras?}

        Esses controles permitem a voc\^e mudar como a vers\~ao para X11 redesenha objetos.
        A caixa de verifica\c{c}\~ao dithering, que somente aparece quando executando em um display de
        oito bits, permite a voc\^e habilitar ou n\~ao a mistura de cores. Dithering \'e o
        m\'etodo pelo qual Geomview utiliza umpequeno conjunto de cores (abaixo de 217) para
        mostrar qualquer cor que voc\^e desejar. Isso \'e feito pela coloca\c{c}\~ao de pixels de pequenas quantidades
        de cores diferentes pr\'{o}ximas umas das outras permitindo que seus olhos harmonize-as
        juntas. Desafortunadamente, isso tem um custo computacional para ser feito.
        Desligando a mistura de cores ir\'a aumantar a velocidade de processamento de imagens, mas as cores usadas n\~ao ir\~ao
        se exatamente o que voc\^e deseja. Dependendo de sua cena, pode ser uma
        pre\c{c}o a pagar aceit\'avel.

        O menu Z-Buffer que aparece permite a voc\^e selecionar entre tr\^es diferentes
        m\'etodos esconder linhas e remover superf\'icie: z-buffering, depth sort, e
        none. Z-buffering is the most accurate and enables the near and far
        clipping planes. Depth sort uses less computing, but will be inaccurate
        if objects intersect (polygons will pop in front when they should be
        partially obscured) and in certain other circumstances (long, narrow
        polygons close to other polygons are one example). Depending on your
        scene, using this method could look just the same as z-buffering but be
        much faster. The ``None'' option turns off all hidden line/surface
        removal. This is only recommended for a scene which consists of just
        lines in one color.

    \subsection{What does ``Not enough colors available. Using private colormap'' mean?}

        This happens when using the X11 version on an eight bit display
        (currently common on workstations). An eight bit display can only show
        256 colors simultaneously. These colors are shared by all the programs
        running. Once a colorcell has been allocated by an application, its
        color is fixed. Geomview tries to grab many colors when it starts. If it
        fails to get them, it prints this message and uses a private colormap. A
        private colormap means that Geomview now has access to all 256
        colorcells. Unfortunately, these colors will only be displayed when the
        cursor is inside one of Geomview's windows. The switching of colormaps
        when the cursor enters and leaves the windows will give a technicolor
        look to the rest of the display.

        If you don't like the technicolor effect, you will have to quit the
        programs which are using up colormap space. Examples of programs which
        use lots of colormap space are background pictures, image viewers,
        visualization software, and WWW browsers.

    \subsection{What does ``Shared memory unavailable, using fallback display method'' mean?}

        The X11 version of Geomview uses the shared memory extension to move
        images quickly between the program and the X server. However, this
        method of communicating with the X server only works when running
        Geomview on the same machine as the display. If Geomview can't use
        shared memory, it prints this message and goes back to using standard X
        calls. Everything will work the same, it will just run much slower,
        especially if you're running over the network.

    \subsection{Why do I get compiler errors about including files Xm/*.h?}

        You're trying to compile the X11 version and the compiler can't find the
        Motif header files. If you have Motif but the headers are in a
        nonstandard place, change the ``SYSCOPTS'' in your
        makefiles/mk.\${MACHTYPE} file. If you don't have Motif, you won't be
        able to compile Geomview. In this case, use one of the binary
        distributions, if you can.

\end{document}
